


\hfill \faUserFriends \barskills{{Relacionamento interpessoal/4.5}}\\
%\vspace{2px}
\hfill \faBalanceScale \barskills{{Ética/5}}\\
%\vspace{2px}
\hfill \faChess \barskills{{Resiliência/4.5}}\\
%\vspace{2px}
\hfill \faUsersCog \barskills{{Resolução de problemas/5}}\\
%\vspace{2px}
\hfill \faCogs \barskills{{Flexibilidade/4}}\\
%\vspace{2px}
\hfill \faDiagnoses \barskills{{Proatividade/4.5}}\\
%\vspace{2px}
\hfill \faUsers \barskills{{Trabalho em equipe/4}}

\begin{commentA} \vspace{0.3cm} \noindent 
\vspace{8px}

\faMedal ~Qualidades de Liderança.\\ \vspace{3px}
\faItchIo ~Capacidade de trabalhar sob pressão. \\ \vspace{3px}
\faLaptopHouse ~Trabalhar de forma independente. \\ \vspace{3px}
\faChalkboardTeacher ~Iniciativa na resolução de problemas. \\ \vspace{3px}
 \par \vspace{0.1cm} \end{commentA}


\begin{commentA} \vspace{0.3cm} \noindent 
Em seu dia a dia, realiza tarefas como:

Desenvolvimento de promoções e campanhas
Apoio para a implementação de ações planejadas
Participa da criação de novos produtos ou serviços
Acompanhamento e gerenciamento de redes sociais
Acompanhamento de métricas e índices de performance
Análise do retorno sobre os investimentos de marketing
Planejamento para aumentar o tráfego orgânico e pago do site
Organização de eventos
Elabora relatórios de resultados
 \par \vspace{0.1cm} \end{commentA}

\begin{commentA} \vspace{0.3cm} \noindent 
Habilidades para Currículo: confira as 14 Competências indispensáveis para sua carreira
Algumas das habilidades indispensáveis para seu currículo são proatividade, boa comunicação, liderança, capacidade de análise, gerenciamento do tempo, inteligência emocional, criatividade, resolução de problemas, além de outras habilidades e competências.

Samanta Jovana
3 jul, 18 | Leitura: 12min
habilidades para currículo
Você já deve ter lido e escutado uma série de dicas sobre como montar seu currículo, que ele não deve ser grande demais, pequeno demais etc. Quer saber a melhor forma de elaborar esse documento? Você precisa focar nas habilidades e competências mais importantes!

Há diversas habilidades para currículo que um profissional pode destacar. Muitas são específicas de cada profissão, como o domínio de alguns softwares é necessário para que alguém se torne um designer ou o pensamento crítico é fundamental para um jornalista. Todavia, há aquelas que servem a vários tipos de profissionais e farão toda a diferença na sua contratação.

Também devem entrar as chamadas soft skills, aquelas características que não podem ser quantificadas, mas representam bem o modo de trabalhar de alguém. Proatividade, boa comunicação e liderança, por exemplo.

Para apresentar as soft skills em um currículo, a melhor forma é por meio do seu histórico profissional e experiências específicas — permitindo que você conte histórias que transmitam essas características.

Confira as 14 competências e habilidades indispensáveis para seu currículo:

Proatividade
Boa comunicação
Relacionamento
Flexibilidade
Liderança
Experiência como freelancer
Autoconfiança
Capacidade de análise
Espírito de equipe
Gerenciamento do tempo
Inteligência emocional
Criatividade e inovação
Capacidade de adaptação
Resolução de problemas
1. Proatividade
A proatividade é dessas características que jamais podem faltar a um profissional. Trata-se da habilidade de antecipar as demandas e anseios das pessoas ao seu redor e contemplá-los sempre que possível. Por isso, é tão bem-vista nos currículos.

Ser proativo significa que você é capaz de resolver problemas e conduzir projetos por conta própria, sem precisar de constante supervisão, o que é uma característica forte das profissões modernas.

Antigamente, os colaboradores tinham papéis muito bem definidos dentro de suas organizações e, deles, não era esperado nada mais do que o cumprimento de suas funções. Hoje, porém, é improvável que você seja contratado para fazer apenas a mesma coisa e não precise de criatividade e senso de prioridade para obter bons resultados.

Profissionais proativos reúnem ambas as características e dão conta do recado, mesmo que não exista alguém pressionando constantemente. Por isso, na hora da contratação, os departamentos de RH colocam esse perfil no topo de suas listas.

2. Boa comunicação
Comunicar-se bem é outro princípio fundamental do mercado de trabalho atual — mas um que já vem de longa data. Saber colaborar com os demais membros da sua equipe e os profissionais de outros departamentos da empresa abrirá portas e fará com que as tarefas assumidas por você sejam mais simples de se completar. Exatamente por esse motivo, a boa comunicação é uma das soft skills favoritas dos recrutadores.

Você pode evidenciá-la logo no primeiro contato com a empresa para a qual deseja trabalhar. Um candidato que se comunica bem sabe redigir e-mails claros, consegue cumprir compromissos e deixa evidentes suas expectativas a respeito da vaga que está prestes a assumir.

Vender-se como um bom comunicador é, então, uma das primeiras tarefas que você assumirá no seu próximo emprego.

Nota do editor:

Aproveite para se 
preparar melhor para uma entrevista de emprego
e confira a videoaula gratuita que contou com a participação das nossas especialistas do RH, 
Sara Cândido
e 
Natália Dantas
.

3. Relacionamento
A capacidade de se relacionar bem com as pessoas é uma habilidade para currículo tão importante que, muitas vezes, aparece como protagonista antes dele.

É por meio de um bom relacionamento que desenvolvemos o networking e conseguimos encontrar oportunidades que tenham a ver com o nosso perfil, por exemplo. Então, dependendo de como você ficou sabendo de uma vaga ou por quem foi indicado, a sua capacidade de relacionamento já estará clara desde o início da conversa.

Caso esse não seja o seu caso, procure mostrar para o seu futuro chefe, nas suas próprias palavras, como você é bom para trabalhar em equipe. Até porque essa também é a melhor estratégia para nutrir essa característica ao longo do tempo.

4. Flexibilidade
Ser flexível é saber trabalhar com os recursos de que se dispõe e não apenas cumprir uma agenda que gira em torno dos momentos em que você é mais produtivo.

Quem cursou uma universidade, provavelmente, já teve que investir nisso, seja porque tinha menos tempo do que imaginava para concluir um projeto, seja porque teve que improvisar com os materiais disponíveis.

O importante na busca por um emprego é deixar isso claro no seu CV ou portfólio. Por isso, seja o mais específico que puder dentro do espaço disponível e mostre os resultados que obteve quando a maré não estava para peixe.

5. Liderança
Liderança é uma daquelas características que não podem faltar. Mas como demonstrá-la em seu currículo e desenvolvê-la ao longo da vida profissional? Em geral, ser líder nos parece sinônimo de “mandar nos outros”, mas vai muito além disso.

Você pode ter assumido um papel de liderança quando coordenou uma equipe ou instruiu os membros do seu departamento na hora de lidar com uma nova tecnologia. E, mesmo que não tenha associado essas atividades ao papel de líder, deverá lembrar-se delas para evidenciar a característica em um currículo.

Se nunca teve alguma experiência do tipo, saiba que nunca é tarde para começar: desenvolva sua liderança investindo em um projeto pessoal e convidando algumas pessoas para participar dele, por exemplo. Esse projeto pode virar algo sério ou, pelo menos, ajudá-lo a construir um portfólio que impressionará futuros empregadores.

6. Experiência como freelancer
Essa é uma habilidade para currículo que consegue condensar várias das soft skills mencionadas até aqui. Isso porque, normalmente, freelancers têm controle completo sobre suas atividades.

Eles precisam desenvolver bons relacionamentos, atualizar constantemente seus conhecimentos, saber adaptar diferentes rotinas de trabalho e ter uma ótima comunicação interpessoal. Sabe por quê?

Porque seus ganhos são diretamente proporcionais à qualidade do seu trabalho e o esforço de promovê-lo. Ou seja, ele precisa ser organizado e ter uma mente empreendedora, sabendo que ele é o responsável por seus resultados, bons ou ruins.

Ser freelancer também é uma forma de se mostrar responsável pelos projetos que se compromete em realizar. Ele tem a autonomia para escolher aqueles com os quais mais se identifica, mas também a sabedoria para usar essa tal liberdade, entende?

Assim, apenas com essa adição em seu currículo, você já conquistou três pontos fundamentais da lista de habilidades para currículo que fizemos aqui!

Porém, não basta incluir a atuação profissional como freelancer para diferenciar-se entre tantos outros profissionais — que, provavelmente, já atuaram assim em algum ponto de suas carreiras. Você deve reforçar essa experiência no currículo com atividades desempenhadas, projetos realizados ou até a menção a alguns clientes satisfeitos.

Verificar que seu trabalho como freelancer foi bem executado fará com que um recrutador consiga ver claramente essas características em você.

7. Autoconfiança
Não poderíamos deixar de falar da autoconfiança, não é mesmo? Não por acaso, um profissional que se apresenta seguro das próprias habilidades profissionais e como uma pessoa capaz de dar conta do recado é muito mais bem-sucedido do que aqueles que não o fazem.

É claro que todos temos um pouco de insegurança, mas, na carreira, devemos fazer o possível para demonstrar que confiamos em nossos talentos e somos capazes de enfrentar desafios.

Pode ser difícil mostrar autoconfiança no começo, porém, mais cedo ou mais tarde, você entenderá que se trata de uma habilidade que se pode treinar. Pense bem nas suas decisões e tente justificá-las com clareza. Quando perceber, já estará fazendo isso como hábito e transparecendo muito mais autoconfiança do que imagina.

8. Capacidade de análise
Analisar, comparar, dar valor aos fatos, sintetizar, organizar as ideias, considerar as alternativas e suas possíveis consequências sem uma supervisão direta também é uma super-habilidade para se ter.

Talvez não seja simples mostrar todo o seu potencial no currículo, é verdade, mas, já percebeu quantas empresas estão empregando testes online? Aliás, eles deveriam vir com uma orientação inicial: deposite aqui toda a sua capacidade analítica e para resolver problemas.

Seja qual for o meio, demonstrar que sabe usar os dados e as variáveis apresentadas em uma situação pode ser uma habilidade profissional determinante.

9. Espírito de equipe
Quando pensamos em trabalho em equipe, acreditamos que ele é o inverso da atividade realizada individualmente. Pronto e acabou, certo? Mas não é bem assim. Existem várias nuances nessas formas de execução.

Só para exemplificar, um dos movimentos da gestão de pessoas é o empowerment, que basicamente, consiste em dar poder de decisão situacional aos funcionários. Legal, né?

Dessa forma, seu colega de trabalho pode ser nomeado o responsável por um projeto em que vocês estejam atuando hoje e, no próximo mês, ser você. Tudo dependerá em quem é mais fera no assunto.

Então, é preciso desmistificar a ideia do chefe e subordinados, e entender que os comandos e ensinamentos podem vir de qualquer nível, desde que seja o melhor para o resultado final.

Ter o espírito de equipe é entender que o resultado alcançado de forma colaborativa é muito melhor, e que todos os incentivos nesse sentido devem ser vivenciados de forma aberta e sem preconceitos.

10. Gerenciamento do tempo
As empresas já entenderam que se manterem enxutas permite que elas estejam muito mais aptas para agarrar novas oportunidades. E, para que elas consigam isso, precisam de funcionários que atuam com alta performance.

Nesse sentido, a alta performance é obtida com o uso máximo das habilidades em todo o tempo útil possível. Assim, um profissional que saiba gerenciar seu tempo e realizar todas as demandas com tranquilidade e qualidade é supervalorizado.

Mas note um detalhe importante: as empresas querem um quadro enxuto de funcionários e alta performance, só que a mistura dessas duas coisas não significa sacrifício ou mil horas extras no final do mês. Essas duas atitudes, inclusive, traduzem situações em que o tempo de trabalho não foi bem administrado, não é mesmo?

Então, aí, mora a diferença, e é nesse ponto que o profissional deve trabalhar sua apresentação no currículo ou até mesmo entrevista de emprego. Descrever, por exemplo, que 90% dos projetos são entregues antes do deadline (prazo final) não é só chique, como também uma forma de demonstrar controle do tempo e alta performance. E tome beijinho no ombro!

11. Inteligência emocional
Essa é uma habilidade muito importante e que pode ser demonstrada no currículo, justamente, valorizando alguns pontos que, normalmente, os candidatos não gostam de mencionar, como ter filhos, ter pouca experiência no mercado em que estão propondo entrar etc.

Essas situações exigem que o profissional saiba lidar com suas emoções e tire o melhor proveito delas para impulsionar suas ações, tanto profissionais quanto pessoais.

Portanto, a inteligência emocional vai girar em torno dos seguintes pontos básicos:

conhecer as próprias emoções, como quais são suas sensações quando está trabalhando para atingir metas, ou como se sente ao saber que está lidando com muitos profissionais de outra geração ou estilo de vida;
controlar as emoções, que consiste em pegar todos esses anseios e estresses, racionalizá-los e, então, encontrar formas de impedir que comprometam sua performance;
automotivar-se;
ter empatia com as emoções dos outros e mostrar-se aberto a ajudar; e
saber se relacionar interpessoalmente.
Isso dará mais controle e percepção dos fatores que desencadeiam seu estresse e, a partir de então, mostrar que sabe gerenciar a vida pessoal e a profissional com total equilíbrio nas duas fará de você um membro confiável e estável do time.

12. Criatividade e inovação
Criatividade, sem dúvidas, é uma habilidade incrível para seu currículo, seja como designer gráfico, seja como psicólogo.

A criatividade também está relacionada à capacidade de trazer soluções inovadoras para problemas ou situações existentes, ou seja, não é apenas para inventar algo novo. Nesse sentido, todo profissional, em diferentes cargos e empresas, precisa saber propor soluções mais eficientes, econômicas, diferenciadas ou o que quer que seja.

Um profissional com essa habilidade não é iluminado, e sim aquele que faz as perguntas certas sobre como resolver um problema, e, claro, também entende a importância de trocar conhecimentos com o restante da equipe para trazer elementos que não são comuns à sua rotina.

É aquela velha história de ver o mesmo problema sob diferentes perspectivas e pontos de vista, sabe?

Se você tem bons exemplos de soluções que propôs em algum projeto ou trabalhos que executou em que conseguiu explorar seus elementos de maneira diferenciada, não deixe de mencioná-los no currículo ou portfólio.

Se pensarmos com a perspectiva de que seu currículo está sendo construído com princípios do marketing de conteúdo, essas menções seriam cases de sucesso, aqueles que valem a pena mencionar para que seu público imagine: “ôh lá em casa, quero dizer, ôh lá na minha equipe, sô!”.

13. Capacidade de adaptação
Ah sim, porque tem essa questão também. Você pode ser um profissional “ôh lá na minha equipe, sô”, mas precisa saber fazer parte dela, se adaptar à cultura da empresa e aos desafios que ela enfrenta.

Se uma empresa funde com a outra, por exemplo, é uma mistura de políticas, processos e elementos afins que precisam se adaptar rapidamente. Se um profissional já teve essa experiência, é ótimo, e ela vai estar claramente descrita no tópico em que as vivências profissionais e seus respectivos períodos estão dispostos.

Mas esse não é o único tipo de adaptação que pode ser mencionada no currículo. Mudanças de sistemas operacionais, por exemplo, podem ser sutilmente mencionadas quando o profissional indica o domínio do uso de dois ou mais deles. Mudanças de cidade para um novo cargo ou emprego, é claro, também dizem tudo.

14. Resolução de problemas
Mencionamos autonomia, criatividade e proatividade, e, para tornar essa mistura ainda melhor no seu currículo, a habilidade para resolver problemas pode ser considerada a cereja do bolo.

Isso porque ela mistura um pouco de cada um desses elementos, como a identificação e criatividade para resolver um problema, com a atitude de se colocar à disposição, tomar à frente. Mas ter um potencial de resolutividade também inclui a consideração de outros aspectos que envolvem o problema e a urgência com que ele precisa ser solucionado.

Muitas vezes, questões meramente hierárquicas podem interferir no processo. E, antes que você tire a conclusão óbvia, nem sempre é culpa do chefe, ou, melhor dizendo, do gestor.

É muito comum que um redator veja uma contestação do cliente em relação a um erro gramatical no texto, e se acomode com o fato de que aquilo seria um problema do responsável pela revisão. Ora, se ele fez o texto, tem o senso de trabalho em equipe e as habilidades profissionais para correção da falta de atenção, por que não corrigir?

Se a sua insegurança estiver relacionada aos seus conhecimentos sobre o como fazer uma boa revisão, por exemplo, ele pode ter a iniciativa de buscar um curso e se aprimorar, certo? Ainda que não seja a solução para aquele problema de imediato, certamente, essa ação trará resultados preventivos valiosos para os seus trabalhos futuros.

Aliás, essa é uma boa ideia, não é mesmo? Então, se inscreva já no curso gratuito de revisão de conteúdo para Web da Rock University.

Saber resolver problemas nos dias de hoje, com milhares de dados disponíveis, clientes exigentes e a supervelocidade das conexões digitais, é imprescindível, assim como todas as demais habilidades.

E, talvez você não tenha todas elas no mais alto grau de performance, é verdade, mas nada impede que desenvolvê-las ou aprimorá-las faça parte de suas metas pessoais e profissionais. Além disso, cada job e emprego exigirá competências e habilidades específicas, o que acaba sendo bem justo, não é mesmo?

Algumas habilidades para currículo são imprescindíveis porque englobam as características que definem um bom profissional. Por isso, elas são comuns a todas as profissões e podem fazer com que você tenha uma carreira mais flexível e consiga atuar em múltiplas áreas sem precisar começar um treinamento do zero.

Fique atento a essas habilidades e comece a desenvolvê-las, afinal, elas representarão um diferencial e tanto na sua jornada.

Agora que você conhece as habilidades para currículo que todo profissional deve ter, que tal aprender a fazer o seu? Veja aqui os melhores modelos de currículo que vão ajudá-lo a completar essa tarefa com mais facilidade!

 \par \vspace{0.1cm} \end{commentA}