%%%%%%%%%%%%%%%%%
% This is an example CV created using altacv.cls (v1.4, 12 Apr 2021) written by
% LianTze Lim (liantze@gmail.com), based on the
% Cv created by BusinessInsider at http://www.businessinsider.my/a-sample-resume-for-marissa-mayer-2016-7/?r=US&IR=T
%
%% It may be distributed and/or modified under the
%% conditions of the LaTeX Project Public License, either version 1.3
%% of this license or (at your option) any later version.
%% The latest version of this license is in
%%    http://www.latex-project.org/lppl.txt
%% and version 1.3 or later is part of all distributions of LaTeX
%% version 2003/12/01 or later.
%%%%%%%%%%%%%%%%

%% Use the "normalphoto" option if you want a normal photo instead of cropped to a circle
% \documentclass[10pt,a4paper,normalphoto]{altacv}

\documentclass[10pt,a4paper,ragged2e,withhyper]{altacv}
%% AltaCV uses the fontawesome5 and simpleicons packages.
%% See http://texdoc.net/pkg/fontawesome5 and http://texdoc.net/pkg/simpleicons for full list of symbols.
%
% Change the page layout if you need to
\geometry{left=1.25cm,right=1.25cm,top=1.5cm,bottom=1.5cm,columnsep=0.8cm}

% The paracol package lets you typeset columns of text in parallel
\usepackage{paracol}

% Change the font if you want to, depending on whether
% you're using pdflatex or xelatex/lualatex
% WHEN COMPILING WITH XELATEX PLEASE USE
% xelatex -shell-escape -output-driver="xdvipdfmx -z 0" sample.tex
\iftutex
% If using xelatex or lualatex:
\setmainfont{Roboto Slab}
\setsansfont{Lato}
\renewcommand{\familydefault}{\sfdefault}
\else
% If using pdflatex:
\usepackage[rm]{roboto}
\usepackage[defaultsans]{lato}
% \usepackage{sourcesanspro}
\renewcommand{\familydefault}{\sfdefault}
\fi

% Defina as cores se quiser aqui
\definecolor{SlateGrey}{HTML}{2E2E2E}
\definecolor{LightGrey}{HTML}{666666}
\definecolor{DarkPastelRed}{HTML}{450808}
\definecolor{PastelRed}{HTML}{8F0D0D}
\definecolor{GoldenEarth}{HTML}{E7D192}

% Minha definição de cores
\definecolor{Mulberry}{HTML}{72243D}
\definecolor{VividPurple}{HTML}{3366CC}
\definecolor{VividPurpleb}{HTML}{3E0097}
% Minha definição de cores para rede sociais
\definecolor{facebook} {HTML}{1877F2}
\definecolor{instagram}{HTML}{E1306C}
\definecolor{tiktok}   {HTML}{000000}
\definecolor{linkedin}  {HTML}{0A66C2}
\definecolor{pinterest} {HTML}{E60023}
\definecolor{twitter}   {HTML}{1DA1F2}
\definecolor{snapchat}  {HTML}{FFFC00}
\definecolor{discord}   {HTML}{7289DA}
% Minha definição de cores para comunicação
\definecolor{whatsapp}{HTML}{25D366}
\definecolor{telegram}{HTML}{0088CC}
\definecolor{slack}{HTML}{4A154B}
\definecolor{skype}{HTML}{00AFF0}
\definecolor{sms}{HTML}{FF6E40}
% Minha definição de cores para streaming
\definecolor{youtube}{HTML}{FF0000}
\definecolor{spotify}{HTML}{1DB954}
\definecolor{vimeo}{HTML}{1AB7EA}
% Minha definição de cores para desing
\definecolor{figma}{HTML}{F24E1E}
\definecolor{invision}{HTML}{FF3366}
\definecolor{behance}{HTML}{1769FF}
\definecolor{envira}{HTML}{6DBE45}
% Minha definição de cores para sistema
\definecolor{salesforce}{HTML}{1798C1}
\definecolor{sass}{HTML}{CC6699}
\definecolor{zendesk}{HTML}{78A300}
\definecolor{jira}{HTML}{0052CC}
\definecolor{atlassian}{HTML}{0052CC}
\definecolor{mailchimp}{HTML}{FFE01B}
% Minha definição de cores para front-end
\definecolor{html5}{HTML}{E44D26}
\definecolor{css3}{HTML}{1572B6}
\definecolor{js}{HTML}{F7DF1E}
\definecolor{bootstrap}{HTML}{7952B3}
\definecolor{ux}{HTML}{00C1D4}
% cores para front-end adicionais
\definecolor{react}{HTML}{61DAFB}      % React
\definecolor{angular}{HTML}{DD0031}    % Angular
\definecolor{vue}{HTML}{4FC08D}        % Vue.js
\definecolor{svelte}{HTML}{FF3E00}     % Svelte
\definecolor{nextjs}{HTML}{000000}     % Next.js
\definecolor{gatsby}{HTML}{663399}     % Gatsby
\definecolor{tailwind}{HTML}{06B6D4}   % Tailwind CSS
\definecolor{typescript}{HTML}{3178C6} % TypeScript
\definecolor{webpack}{HTML}{8DD6F9}    % Webpack
\definecolor{babel}{HTML}{F9DC3E}      % Babel
\definecolor{jest}{HTML}{C21325}       % Jest
\definecolor{cypress}{HTML}{04B38D}    % Cypress
\definecolor{storybook}{HTML}{FF4785}  % Storybook
\definecolor{graphql}{HTML}{E10098}    % GraphQL
\definecolor{rest}{HTML}{6C7A89}       % REST APIs (cinza-azulado)
% Minha definição de cores para back-end
\definecolor{python}{HTML}{3776AB}
\definecolor{node}{HTML}{339933}
\definecolor{docker}{HTML}{2496ED}
\definecolor{github}{HTML}{181717}
\definecolor{markdown}{HTML}{083FA1}
% Minha definição de cores para banco de dados
\definecolor{postgresql}{HTML}{336791}
\definecolor{mysql}{HTML}{00758F}
% Minha definição de cores para site
\definecolor{wordpress}{HTML}{21759B}
\definecolor{wix}{HTML}{2D00F7}
\definecolor{cpanel}{HTML}{EA5504}
\definecolor{joomla}{HTML}{ED1C24}
\definecolor{drupal}{HTML}{0C76AB}
\definecolor{blogger}{HTML}{FB8F3D}
% Minha definição de cores para nuvem
\definecolor{aws}{HTML}{FF9900}
\definecolor{server}{HTML}{999999}
\definecolor{dropbox}{HTML}{0061FF}
\definecolor{googledrive}{HTML}{4285F4}
% Infraestrutura pública
\definecolor{aws}{HTML}{FF9900}        % já definido
\definecolor{azure}{HTML}{0078D4}      % Microsoft Azure
\definecolor{gcp}{HTML}{4285F4}        % Google Cloud
\definecolor{ibmcloud}{HTML}{054ADA}   % IBM Cloud (azul escuro)
\definecolor{oraclecloud}{HTML}{F80000}% Oracle Cloud (vermelho)
\definecolor{alibabacloud}{HTML}{FF6A00}% Alibaba Cloud (laranja)
\definecolor{digitalocean}{HTML}{0080FF}% DigitalOcean (azul)
\definecolor{heroku}{HTML}{6762A6}     % Heroku (roxo)
\definecolor{cloudflare}{HTML}{F38020} % Cloudflare (laranja)
\definecolor{openstack}{HTML}{00B1E1}  % OpenStack (azul claro)

% Aplique as cores dos componentes de texto
\colorlet{name}{black}
%\colorlet{tagline}{PastelRed}
%\colorlet{heading}{DarkPastelRed}
%\colorlet{headingrule}{GoldenEarth}
%\colorlet{subheading}{PastelRed}
%\colorlet{accent}{PastelRed}
\colorlet{emphasis}{SlateGrey}
\colorlet{body}{LightGrey}
% My colours definition
\colorlet{tagline}{DarkPastelRed}
\colorlet{heading}{VividPurple}
\colorlet{headingrule}{VividPurple}
\colorlet{subheading}{VividPurple}
\colorlet{accent}{VividPurple}

% Change some fonts, if necessary
\renewcommand{\namefont}{\Huge\rmfamily\bfseries}
\renewcommand{\personalinfofont}{\footnotesize}
\renewcommand{\cvsectionfont}{\large\rmfamily\bfseries}
\renewcommand{\cvsubsectionfont}{\large\bfseries}


% Change the bullets for itemize and rating marker
% for \cvskill if you want to
\renewcommand{\cvItemMarker}{{\small\textbullet}}
\renewcommand{\cvRatingMarker}{\faCircle}
% ...and the markers for the date/location for \cvevent
% \renewcommand{\cvDateMarker}{\faCalendar*[regular]}
% \renewcommand{\cvLocationMarker}{\faMapMarker*}


% If your CV/résumé is in a language other than English,
% then you probably want to change these so that when you
% copy-paste from the PDF or run pdftotext, the location
% and date marker icons for \cvevent will paste as correct
% translations. For example Spanish:
% \renewcommand{\locationname}{Ubicación}
% \renewcommand{\datename}{Fecha}


%% Use (and optionally edit if necessary) this .tex if you
%% want to use an author-year reference style like APA(6)
%% for your publication list
% \input{pubs-authoryear.cfg}

%% Use (and optionally edit if necessary) this .tex if you
%% want an originally numerical reference style like IEEE
%% for your publication list
%\input{pubs-num.cfg}

%% sample.bib contains your publications
%\addbibresource{sample.bib}

\usepackage[author={David N da Silva}]{pdfcomment}
\usepackage{dashrule}  % no preâmbulo, já deve estar carregado
\usepackage{multicol}
% opcional: distância entre colunas
\setlength{\columnsep}{1em}

% No preâmbulo, depois de carregar altacv.cls:
%\setlength{\smallskipamount}{2pt}
\setlength{\medskipamount}{4pt}

\makeatletter
\renewcommand*{\divider}{%
	\par
	\vspace{-0.8em}% espaço acima
	\noindent
	\hfill
	{\color{headingrule}% opcional: usa cor de headingrule
	\hdashrule{\linewidth}{0.5pt}{1mm 1mm}}\par% risquinho tracejado de 2cm
	\vspace{2pt}% espaço abaixo
}
\makeatother

\begin{document}

\Aoff
\Bon
\Con	

\name{David Nascimento da Silva}
\tagline{%-------------------------
% Profissões
Especialista Inteligência Marketing Growth, Digital e Comercial
%Analista de Marketing Bilingue
%---------------------------------------------------------------------------
%Júnior-------------
%É o nível dos profissionais mais inexperientes, recém-formados, que estão ingressando no mercado profissional. Por conta disso, recebem tarefas de menor complexidade. Isso não significa que o profissional deve agir com irresponsabilidade. Pelo contrário. É imprescindível assumir as suas funções com comprometimento para se destacar e evoluir na carreira.
%Complexidade das funções: baixo 
%Formação: recém-formado 
%Pleno -------------
%Conforme adquire experiência, o profissional ganha confiança para exercer funções com maior grau de complexidade e que exigem mais conhecimento técnico da sua área de atuação. Além disso, tem um pouco mais de liberdade para tomar decisões.
%Complexidade das funções: médio
%Formação: especialização, MBA
%Sênior-------------
%O profissional sênior possui tarefas altamente complexas, tem autonomia para tomar decisões e alto grau de maturidade e inteligência emocional. Seus conhecimentos técnicos são mais aprofundados, além de possuir habilidades comunicacionais, de gestão e liderança.
%Complexidade das funções: alta
%Formação: especialização, MBA, mestrado profissional 
%Experiência: mais de 10 anos }
%% You can add multiple photos on the left or right
\photoR{3cm}{foto2.jpg}
% \photoL{2cm}{Yacht_High,Suitcase_High}

\personalinfo{%
	%-------------------------
% Cabeçalho
% Not all of these are required!
% You can add your own with \printinfo{symbol}{detail}
\email{davidnascimentodasilva@gmail.com*}
\phone{85 99912-5992}
\mailaddress{Rua Silva Paulet 776, Apto. 202 - Aldeota}
\location{Fortaleza, CE}
\habilitacao{A \& B}
%\homepage{davidsilva.tumblr.com}
\twitter{@Dolfino*}
\linkedin{davidnsilva*}
\github{github.com/Dolfino*}% I'm just making this up though.
\textbf{Clique!\pdfcomment[color=blue,icon=Comment]{Mais informações: Caso precise de um portfólio}}
%\orcid{0000-0000-0000-0000} % Obviously making this up too.
%% You can add your own arbitrary detail with
%% \printinfo{symbol}{detail}[optional hyperlink prefix]
%\Behance{Dolfino}
%\printinfo{\faPaw}{Hey ho!}
	\cvsubsection{RESUMO\hfill}
	%-------------------------
% Resumo
%\begin{commentA} \vspace{0.3cm} \noindent 
Especialista em Inteligência de Marketing com foco em Growth Marketing, análise de dados e otimização de campanhas digitais. Experiência sólida no desenvolvimento de estratégias inorgânicas e orgânicas de aquisição de tráfego, integrandoferramentas de marketing digital, como Google BigQuery/SQL, Google Ads e Meta Ads. Habilidade em traduzir dados complexos em estratégias acionáveis que impulsionam o crescimento sustentável, com uma abordagem orientada para resultados mensuráveis.
%\par \vspace{0.1cm} \end{commentA}
\begin{commentA} \vspace{0.3cm} \noindent 
Programador
Programador dedicado com 5 anos de experiência. Tenho experiência em Java, Python e C#, programando diversos tipos de aplicações para os clientes da Empresa X, desde aplicativos bancários até softwares educativos. Focando em otimizar processos, consegui reduzir o tempo de testes dos produtos em 20\%, sem comprometer a qualidade final. Na sua empresa, buscarei oportunidades semelhantes para otimizar processos.
\par \vspace{0.1cm} \end{commentA}
\begin{commentA} \vspace{0.3cm} \noindent 
Recepcionista
Recepcionista bilíngue com 4 anos de experiência. Fluente em Inglês, passei esses 4 anos como recepcionista na Empresa X, atendendo os clientes internacionais da empresa. Em 2018 e 2019, recebi um bônus, pois a minha proficiência em informática me permitiu simplificar processos e aumentar a minha produtividade em 30\% em relação aos anos anteriores. Busco a oportunidade de conseguir resultados ainda melhor na Empresa Y.
\par \vspace{0.1cm} \end{commentA}
\begin{commentA} \vspace{0.3cm} \noindent 
Designer
Designer gráfico inovador com 10 anos de experiência. Ao longo da minha carreira, criei peças publicitárias para empresas multinacionais, como a Empresa X e a Empresa Y. As minha peças ganharam o Prêmio X em 2017 e o Prêmio Y em 2019. Planejo usar as minhas técnicas vencedoras para trazer prêmios para a sua agência, aumentando ainda mais a exposição das peças criadas.
\par \vspace{0.1cm} \end{commentA}
\begin{commentA} \vspace{0.3cm} \noindent 
Editor de textos
Editor de texto eficaz com 5 anos de experiência. Durante esses anos, eu editei textos para diversas plataformas, incluindo blogs, websites e jornais. Com muito foco, consegui reduzir o tempo de produção de cada artigo em 50\%. O meu desejo é continuar encontrando formas de produzir ainda mais rápido, sem perder a qualidade, e é com a Empresa X que pretendo conseguir isso.
\par \vspace{0.1cm} \end{commentA}
	%% You can add your own arbitrary detail with
	%% \printinfo{symbol}{detail}[optional hyperlink prefix]
	% \printinfo{\faPaw}{Hey ho!}[https://example.com/]
	
	%% Or you can declare your own field with
	%% \NewInfoFiled{fieldname}{symbol}[optional hyperlink prefix] and use it:
	% \NewInfoField{gitlab}{\faGitlab}[https://gitlab.com/]
	% \gitlab{your_id}
	%%
	%% For services and platforms like Mastodon where there isn't a
	%% straightforward relation between the user ID/nickname and the hyperlink,
	%% you can use \printinfo directly e.g.
	% \printinfo{\faMastodon}{@username@instace}[https://instance.url/@username]
	%% But if you absolutely want to create new dedicated info fields for
	%% such platforms, then use \NewInfoField* with a star:
	% \NewInfoField*{mastodon}{\faMastodon}
	%% then you can use \mastodon, with TWO arguments where the 2nd argument is
	%% the full hyperlink.
	% \mastodon{@username@instance}{https://instance.url/@username}
}

\makecvheader
%% Depending on your tastes, you may want to make fonts of itemize environments slightly smaller
\AtBeginEnvironment{itemize}{\small}

%% Set the left/right column width ratio to 6:4.
\columnratio{0.6}

% Start a 2-column paracol. Both the left and right columns will automatically
% break across pages if things get too long.
\begin{paracol}{2}

%------------------------------------------------------------------------------------------
\cvsection{ \faGraduationCap ~FORMAÇÃO ACADÊMICA}
%-------------------------
% Formação Acadêmica
\cvevent
{\faUserGraduate\ Bacharel em Marketing\ \hfill \faMedal}
{{\raisebox{-1ex}{\includegraphics[height=4ex]{fbuni.png}}\ Centro Universitário Farias Brito}
	\hfill \faCheck\ Concluído}
{Agosto 2017 -- Junho 2021}
{Fortaleza, CE, Brasil}
\begin{commentA} \vspace{0.3cm} \noindent 
\vspace{5px}
%-------------------------
% Evento 2
\cvevent
{\faUserGraduate Pós-graduação em Gestão de Negócios e Marketing}
{ESPM Escola Superior de Propaganda e Marketing\hfill Estudando}
{Ago/2021 -- Atual}
{Fortaleza, CE, Brasil}
\vspace{5px}
%-------------------------
% Evento 3
\cvevent
{\faUserGraduate Graduado em Administração}
{Universidade de Federal do Ceará \hfill \faCalculator Rank: 04/60}
{janeiro 1997 -- dezembro 2001}
{Fortaleza, CE, Brasil}
\par \vspace{0.1cm} 
\end{commentA}

%------------------------------------------------------------------------------------------
\cvsection{\faIdBadge[regular] EXPERIÊNCIA PROFISSIONAL}
%-------------------------
% Experiência Profissional

% Evento 1
\cvevent
{\faPortrait\ Head em Growth Marketing\ \hfill \faAddressCard}
{\faDiscourse\ Somapay Banco Digital \hfill Último trabalho: 8 meses}
{Novembro 2024 -- Junho 2025}
{Fortaleza, CE, Brasil}
% Ajusta espaçamento e remove “corrida” de itens
\begin{itemize}[leftmargin=*,itemsep=0.5em,topsep=0.5em]
	\item Analisa o contexto do mercado onde está inserido para desenvolver planos de ações estratégicos, estudando os públicos-alvo do produto ou serviço com o intuito de ampliar a base de clientes e fidelizar os já existentes.
	\item Participa da criação e lançamento de novos produtos e serviços, além de definir padrões, identidade visual e posicionamento da marca.
\end{itemize}
\divider
%-------------------------
% Evento 2
\cvevent
{\faChartLine\ Especialista em Growth Marketing\ \hfill \faAddressCard}
{\faQuinscape\ Alloha Fibra \hfill 1 ano}
{Outubro 2023 -- Outubro 2024}
{Fortaleza, CE, Brasil}
% Ajusta espaçamento e remove “corrida” de itens
\begin{itemize}[leftmargin=*,itemsep=0.5em,topsep=0.5em]
	\item Liderança no desenvolvimento de estratégias aquisição multicanal,
	resultando em um aumento de 80\% no tráfego em 2 meses por meio de
	SEO, campanhas de Google Ads e Meta Ads.
	\item Colaboração com as equipes de produto e UX para otimizar a jornada do
	cliente, aumentando o Lifetime Value (LTV) e reduzindo o churn.
\end{itemize}
\divider
%-------------------------
% Evento 3
\cvevent
{\faHandshake[regular]\ Gestão de Relacionamento com o Cliente (CRM)\ \hfill \faAddressCard}
{\faQuinscape\ Alloha Fibra \hfill 8 meses}
{Fevereiro 2023 -- Outubro 2023}
{Fortaleza, CE, Brasil}
% Ajusta espaçamento e remove “corrida” de itens
\begin{itemize}[leftmargin=*,itemsep=0.5em,topsep=0.5em]
	\item Desenvolvi e implementei estratégias de CRM, segmentando a base de
	clientes com base em comportamento e preferências, o que resultou em
	um aumento de 80\% na retenção de clientes.
	\item Desenvolvi relatórios de performance do ciclo de vida do cliente, incluindo
	métricas como Customer Lifetime Value (CLV) e taxa de churn.
\end{itemize}
\divider
%-------------------------
% Evento 4
\cvevent
{\faDiagnoses\ Especialista Marketing Digital \& Analytics\ \hfill \faAddressCard}
{\faQuinscape\ Alloha Fibra \hfill 1 ano}
{Janeiro 2022 -- Fevereiro 2023}
{Fortaleza, CE, Brasil}
% Ajusta espaçamento e remove “corrida” de itens
% Ajusta espaçamento e remove “corrida” de itens
\begin{itemize}[leftmargin=*,itemsep=0.5em,topsep=0.5em]
	\item Gerenciamento e otimização de campanhas mídia paga em Google Ads e
	Meta Ads, com foco na redução de custos e maximização de retorno sobre
	investimento (ROI).
	\item Desenvolvimento de relatórios detalhados sobre performance de campanhas, utilizando ferramentas como Google Analytics e SEMrush, para ajustes em tempo real, resultando em um aumento de 50\% no engajamento.
\end{itemize}
\begin{commentA} \vspace{0.3cm} \noindent 
%-------------------------
% Evento 5
\cvevent
{\faPortrait\ Analista de Marketing Bilíngue}
{\faUber\ Uber do Brasil \hfill Último trabalho: 4,1 anos}
{Abril 2017 -- Maio 2021}
{Fortaleza, CE, Brasil}
% Ajusta espaçamento e remove “corrida” de itens
\begin{itemize}[leftmargin=*,itemsep=0.5em,topsep=0.5em]
	\item Analisa o contexto do mercado onde está inserido para desenvolver planos de ações estratégicos, estudando os públicos-alvo do produto ou serviço com o intuito de ampliar a base de clientes e fidelizar os já existentes.
	\item Participa da criação e lançamento de novos produtos e serviços, além de definir padrões, identidade visual e posicionamento da marca.
	\item Gerencia as redes sociais e blogs de conteúdo, produzindo e programando posts alinhados à estratégia de marketing.
	\item Acompanha métricas e índices de performance para avaliar a eficácia das ações e o retorno sobre o investimento.
\end{itemize}
\divider
%-------------------------
% Evento 6
\cvevent
{\faShoppingBag\ Assistente de Relacionamento Bilíngue}
{\faShoppingBag\ RioMar Shopping Fortaleza S.A \hfill 2,1 anos}
{Outubro 2014 -- Novembro 2016}
{Fortaleza, CE, Brasil}
% Ajusta espaçamento e remove “corrida” de itens
\begin{itemize}[leftmargin=*,itemsep=0.5em,topsep=0.5em]
	\item Recepcionava e prestava suporte a clientes, esclarecendo dúvidas e solucionando problemas.
	\item Desenvolvia aplicativos customizados em Excel para geração de indicadores de desempenho.
	\item Elaborava contratos e coordenava a prestação de serviços com fornecedores.
\end{itemize}
\divider
%-------------------------
% Evento 7
\cvevent
{\faPortrait ~Gerente de Hotel}
{\faHotel Itaca Hotéis Ltda \hfill 4.5 anos}
{Janeiro 2010 --  Junho 2014}
{Caucaia, CE, Brasil}
% Ajusta espaçamento e remove “corrida” de itens
\begin{itemize}[leftmargin=*,itemsep=0.5em,topsep=0.5em]
	\item Coordenando o trabalho de recepção, compra e setor de A\&B;
	\item Coordena os serviços prestados aos clientes VIP;
	\item Supervisionando os serviços do RH, auxiliando e apoiando as atividades sociais dos hóspedes.
\end{itemize}
Contato: Angel Cecilio Lasala Perez 85 985847264
 \par \vspace{0.1cm}
\end{commentA}



%------------------------------------------------------------------------------------------
%\cvsection{\faLanguage ~IDIOMAS}
%%-------------------------
% Idioma
\cvskill{\brasil Português}{5}
\cvskill{\usa Inglês}{4}
\cvskill{\espanha Espanhol}{4}
\cvskill{\holanda Holandês}{3}

%------------------------------------------------------------------------------------------
\newpage
%------------------------------------------------------------------------------------------
\cvsection{\faCertificate CERTIFICAÇÕES}
%-------------------------
% Certificações

\cvevent
	{%
		\faGoogle\ Certificações Google ADS%
		\quad% pequeno espaço antes da linha
		\hrulefill%
		}
	{Google Skillshop \hfill \faEdge Treinamento Web}
	{Julho 2023 -- Agosto 2023}
	{Fortaleza, CE, Brasil}
% remove aquele espaço extra logo após o título
\vspace{-0.5em} 
% Ajusta espaçamento e remove “corrida” de itens
\begin{multicols}{2}
	\begin{itemize}[leftmargin=*,itemsep=0.5em,topsep=0.5em]
		\item Anúncios de a performance com tecnologia de IA.
		\item Aumente as vendas off-line.
		\item Rede de pesquisa do Google Ads.
		\item Criativos do Google Ads.
		\item Anúncios do Shopping.
		\item Apps do Google Ads.
		\item Campanhas	de vídeo no Google Ads.
		\item Display do Google Ads.
		\item Certificação em métricas do Google Ads.
	\end{itemize}
\end{multicols}

\cvevent
	{\faFacebook  Certificações Meta ADS%
		\quad% pequeno espaço antes da linha
		\hrulefill%
		}
	{Meta Blueprint \hfill \faEdge Treinamento Web}
	{Agosto 2023 -- Setembro 2023}
	{Fortaleza, CE, Brasil}
% remove aquele espaço extra logo após o título
\vspace{-0.5em} 
% Ajusta espaçamento e remove “corrida” de itens
\begin{multicols}{2}
	\begin{itemize}[leftmargin=*,itemsep=0.5em,topsep=0.5em]
		\item Planeiamento de Mídia.
		\item Compra de Mídia.
		\item Marketing Science.
		\item Estratégia Criativa.
		\item Associado de Marketing Digital.
		\item Gerenciamento de Comunidade.
		\item Estratégia de Marketing de Negócios.
	\end{itemize}
\end{multicols}

\cvevent
{\faMicrosoft Certificados Microsoft Office Specialist Master}
{Microsoft \hfill \faEdge Treinamento Web}
{Junho 2019 -- Agosto 2019}
{Fortaleza, CE, Brasil}
\cvtag
{\faFileExcel[regular] Excel 2019 Expert} 
\cvtag
{\faFileWord[regular] Word 2019 Expert} 
\cvtag
{Outlook 2019} \\ 
\cvtag
{\faDatabase Access 2016} 
\cvtag
{\faFilePowerpoint[regular] PowerPoint 2016}  
\cvtag
{Project 2019}

%------------------------------------------------------------------------------------------
\cvsection{\faUserCog ~CURSOS COMPLEMENTARES}
%-------------------------
% Cursos Complementares
%\cvevent{Workshop Santander bootcamp}{Digital Innovation One}{Maio 2021 -- Junho 2021}{}
%\vspace{5px}
\cvevent{Curso completo de Business Intelligence com SQL Server}{Udemy \hfill \faStopwatch 120 horas}{Outubro 2024 -- Novembro 2024}{}
\divider
\vspace{1px}
\cvevent{Digital Media and Marketing Strategies}{Coursera \hfill \faStopwatch 60 horas}{Junho 2024 -- Agosto 2024}{}
  
%------------------------------------------------------------------------------------------
\cvsection{\faGlobeAmericas ~EXPERIÊNCIA INTERNACIONAL}



\cvevent{Analista de dados}{MediaMarkt Nederland \hfill 2.2 anos}{Agosto 2007 -- Outubro 2009}{Rotterdam, Netherlands}
\begin{itemize}
	\item Empresa: Varejo eletro eletrônico. 
\end{itemize}

\divider

\cvevent{Analista de suporte}{Philips Benelux \hfill 2.1 anos}{Julho 2005 -- Agosto 2007}{’s-Hertogenbosch, Netherlands}
\begin{itemize}
	\item Empresa: Máquinas e equipamentos
\end{itemize}
	
%------------------------------------------------------------------------------------------
\cvsection{\faAddressBook[regula] COMPETÊNCIAS PESSOAL}



%% Adapted from @Jake's answer from http://tex.stackexchange.com/a/82729/226
%% \wheelchart{outer radius}{inner radius}{
%% comma-separated list of value/text width/color/detail}
%% Some ad-hoc tweaking to adjust the labels so that they don't overlap
\hspace*{-1em}  %% quick hack to move the wheelchart a bit left
\wheelchart{1.5cm}{0.5cm}{%
	25/12em/accent!60/Resolução de problemas,
	10/11em/accent!30/Trabalho em equipe,
	20/11em/accent!40/Proatividade,
	5/ 8em/accent!10/Extrovertido,
	5/ 8em/accent!20/Analítico,
	5/ 8em/accent!20/Ética,
	30/14em/accent/Relacionamento\\Interpessoal
}


%------------------------------------------------------------------------------------------
%\cvsection{Publications}
%\input{Conteúdo/Publicações.tex}
  
%------------------------------------------------------------------------------------------
%% Switch to the right column. This will now automatically move to the second
%% page if the content is too long.
\switchcolumn
%------------------------------------------------------------------------------------------
%\cvsection{Life Philosophy}
\begin{quote}


``If you don't have any shadows, you're not standing in the light.``

\begin{commentA} \vspace{0.3cm} \noindent
“Só existem dois dias no ano que nada pode ser feito.
Um se chama ontem e o outro se chama amanhã, portanto
hoje é o dia certo para amar, acreditar, fazer e principalmente
viver.“
\hfill \faHamsa Dalai Lama
 \par \vspace{0.1cm} \end{commentA}
\end{quote}

%------------------------------------------------------------------------------------------
\cvsection{\faAward HABILIDADES EXPERIÊNCIA}
%-------------------------
%Habilidade e Experiência
\begin{itemize}[nosep,
	labelwidth=1.8em,   % largura reservada pro ícone
	labelsep=0.5em,     % espaço entre ícone e texto
	leftmargin=!,       % alinha tudo na mesma indentação
	itemsep=3pt]        % controla o espaçamento vertical
	\item[\faArchive]      CRM de Marketing da \faSalesforce\ Salesforce
	\item[\faDesktop]      Operação avançadas tickets Zendesk
	\item[\faCommentsDollar]  Ações de CRM de curto e longo prazo.
	\item[\faMailBulk]     Relacionamento em e-mail marketing.
	\item[\faHockeyPuck]   Importar ou vincular banco de dados SQL.
	\item[\faMagic]        Análise com softwares estatísticos IBM-SPSS, Tableau e outros.
	\item[\faMapSigns]     Elaborar e averiguar os dados/relatórios de desempenho de campanhas.
	\item[\faGooglePlusG]  Dados do Google Analytics.
	\item[\faInbox]        Ferramentas de e-mail MKT.
	\item[\faLaptopHouse]  Plano de marketing.
	\item[\faHandHoldingUsd]  Análise de Negócios.
	\item[\faHandsWash]    Implementação e gestão de Projetos.
	\item[\faHeadSideCough]  Gestão da relação com o cliente.
	\item[\faHighlighter]  Desenvolvimento de Conteúdo / Escrita.
\end{itemize}


%------------------------------------------------------------------------------------------
\cvsection{\faUserTie ~COMPETÊNCIAS PROFISSIONAL}
%-------------------------
% Competência Profissional
%\begin{commentA} \vspace{0.3cm} \noindent 
\begin{itemize}[nosep,
	labelwidth=1.8em,
	labelsep=0.5em,
	leftmargin=!,
	itemsep=3pt]
	\item[\faGoogle]           Expertise em Google Analytics.
	\item[\faFacebook]         Facebook Adsense.
	\item[\faChartBar]         Compreensão das estratégias de campanha e práticas de mercado.
	\item[\faBarcode]          Elaborar e averiguar os dados/relatórios de desempenho de campanhas.
	\item[\faFunnelDollar]     Análise do funil de venda e trajetória do cliente.
	\item[\faCoins]            Medição do ROI de campanhas de publicidade on-line e off-line.
	\item[\faCopy]    Elaborar relatórios sobre KPIs de campanhas de marketing, com clientes potenciais, taxas de conversão, tráfego de sites e engajamento nas mídias sociais.
	\item[\faTrafficLight]     Monitorar a distribuição do orçamento e o desempenho de campanhas publicitárias pagas.
\end{itemize}
%\par \vspace{0.1cm} \end{commentA}

\cvsection{\faLanguage ~IDIOMAS}
%-------------------------
% Idioma
\cvskill{\brasil Português}{5}
\cvskill{\usa Inglês}{4}
\cvskill{\espanha Espanhol}{4}
\cvskill{\holanda Holandês}{3}
%------------------------------------------------------------------------------------------
\cvsection{\faLaptop ~COMPETÊNCIAS EM SOFTWARE}
%-------------------------
% Competência Software
\cvsubsection{\faFacebookSquare Rede Social:}
\cvtag[facebook]{\faFacebook\ FaceBook}
\cvtag[instagram]{\faInstagramSquare\ Instagram}
\cvtag[tiktok]{\includegraphics[height=1.5ex]{tiktok.jpg}\ Tiktok}
\cvtag[linkedin]{\faLinkedin\ Linkedin}
\cvtag[pinterest]{\faPinterest\ Pinterest}
\cvtag[twitter]{\faTwitterSquare\ Twitter}
\cvtag[snapchat]{\faSnapchat\ Snapchat}
\cvtag[discord]{\faDiscord\ Discord}
\vspace{5px}
\divider

\cvsubsection{\faWhatsappSquare\ Comunicação:}
\cvtag[whatsapp]{\faWhatsapp\ Whatsapp}
\cvtag[telegram]{\faTelegram\ Telegram}
\cvtag[slack]{\faSlack\ Slack}
\cvtag[skype]{\faSkype\ Skype}
\cvtag[sms]{\faSms\ Sms}
\vspace{5px}
\divider


\cvsubsection{\faStream\ Streaming:}
\cvtag[youtube]{\faYoutube\ Youtube}
\cvtag[spotify]{\faSpotify\ Spotify}
\cvtag[vimeo]{\faVimeo\ Vimeo}
\vspace{5px}
\divider

\cvsubsection{Desing:}
\cvtag[figma]{\faFigma\ Figma}
\cvtag[invision]{\faInvision\ Invision}
\cvtag[behance]{\faBehance\ Behance}
\cvtag[envira]{\faEnvira\ Envira}
\vspace{5px}
\divider

\cvsubsection{\faNetworkWired\ Sistema:}
\cvtag[salesforce]{\faSalesforce\ Salesforce}
\cvtag[sass]{\faSass\ SAS}
\cvtag[zendesk]{\faWindowRestore\ ZenDesk}
\cvtag[jira]{\faJira\ Jira}
\cvtag[atlassian]{\faAtlassian\ Atlassian}
\cvtag[mailchimp]{\faMailchimp\ Mailchimp}
\vspace{5px}
\divider

% --- Front-end ------------------------------------------------
\cvsubsection{\faFileCode\ Front-end:}
\cvtag[html5]{\faHtml5\ HTML5}
\cvtag[css3]{\faCss3\ CSS3}
\cvtag[js]{\faJs\ JavaScript}
\cvtag[typescript]{\faFileCode\ TypeScript}
\cvtag[react]{\faReact\ React}
\cvtag[angular]{\faAngular\ Angular}
\cvtag[vue]{\faVuejs\ Vue.js}
\cvtag[nextjs]{Next.js}
\cvtag[tailwind]{Tailwind CSS}
\cvtag[sass]{Sass/SCSS}
\cvtag[bootstrap]{\faBootstrap\ Bootstrap}
\cvtag[webpack]{Webpack}
\cvtag[babel]{Babel}
\cvtag[graphql]{GraphQL}
\cvtag[rest]{REST APIs}
\vspace{5pt}
\divider

\cvsubsection{\faCode\ Back-end:}
\cvtag[python]{\faPython\ Python}
\cvtag[node]{\faNode\ Node}
\cvtag[docker]{\faDocker\ Docker}\\
\cvtag[github]{\faGithub\ GitHub}
\cvtag[markdown]{\faMarkdown\ Markdown}
\vspace{5px}
\divider

\cvsubsection{\faDatabase\ Banco de dados:}
\cvtag[postgresql]{\faDeskpro\ PostgreSQL}
\cvtag[mysql]{\faDatabase\ MySQL}
\vspace{5px}
\divider

\cvsubsection{\faTable\ Site:}
\cvtag[wordpress]{\faWordpress\ Wordpress}
\cvtag[wix]{\faWix\ Wix}
\cvtag[cpanel]{\faCpanel\ Cpanel}
\cvtag[joomla]{\faJoomla\ Joomla}
\cvtag[drupal]{\faDrupal\ Drupal}
\cvtag[blogger]{\faBlogger\ Blogger}
\vspace{5px}
\divider

\cvsubsection{\faSitemap\ Nuvem:}
\cvtag[aws]{\faAws\ AWS}
\cvtag[server]{\faServer\ Server}
\cvtag[dropbox]{\faDropbox\ Dropbox}\\
\cvtag[googledrive]{\faGoogleDrive\ GoogleDrive}
\cvtag[aws]{\faAws\ AWS}
\cvtag[azure]{\faMicrosoft\ Azure}
\cvtag[gcp]{\faGoogle\ GCP}
\cvtag[oraclecloud]{\ Oracle Cloud}
\cvtag[digitalocean]{\faDigitalOcean\ DigitalOcean}
\cvtag[heroku]{\faEthereum \ Heroku}
\cvtag[cloudflare]{\faCloudflare\ Cloudflare}

\begin{commentA} \vspace{0.3cm} \noindent 
	\cvsubsection{\faDev Programação:}
	\cvtag{\LaTeX}
	\cvtag{\faLeaf OverLeaf}
	\cvtag{\faRProject R Project}
	
	\vspace{4px}
	\par \vspace{0.1cm} \end{commentA}

\begin{commentA} \vspace{0.3cm} \noindent 
\cvsubsection{\faUsers Organização:}
\cvtag{\faTrello Trello}
\cvtag{\faEvernote Evernote}
\cvtag{\faGetPocket Pocket}

\divider
\vspace{1px}
\par \vspace{0.1cm} \end{commentA}

\begin{commentA} \vspace{0.3cm} \noindent 
\cvsubsection{\faLaptopCode Sistema Operacional:}
\cvtag{\faLinux Linux}
\cvtag{\faSafari Safari}
\cvtag{\faWindows Windows}

\divider
\vspace{1px}
\par \vspace{0.1cm} \end{commentA}

\begin{commentA} \vspace{0.3cm} \noindent 
\cvsubsection{\faMobile* Mídia Móvel:}
\cvtag{\faWaze Waze}

\vspace{4px}
\par \vspace{0.1cm} \end{commentA}



\begin{commentA} \vspace{0.3cm} \noindent 

\cvsubsection{ Loja de aplicativos:}
\cvtag{\faGooglePlay GooglePlay}
\cvtag{\faAppStore AppStore}
\cvtag{\faApplePay ApplePay}

\cvtag{\faCode}
\cvtag{\faCloudUpload*}

\cvtag{\faCoffee}
\cvtag{\faCopyright[regular]}
\cvtag{\faDatabase}
\cvtag{\faJenkins}



\faFunnelDollar
\faTelegramPlane
\faDev
\faFileCode
\faFileCode[regular]
\faFilePdf[regular]
\faFileSignature
\faApple
\faRetweet
\faItunes

\cvtag{\faMicrosoft Microsoft}
\cvtag{\faBlackberry Blackberry}




\faGlasses
\faHashtag
\faHome
\faHouseUser

\faLaptopHouse




\faMarker

\faMedium
\faMediumM
\faMendeley
\faMobile

\faMoneyBill*[regular]

\faNewspaper[regular]
\faPatreon
\faPaste
\faPaypal


\faReadme
\faReddit
\faRedditAlien
\faRedditSquare
\faRssSquare
\faRss
\faScribd
\faSellsy
\faShopify


\faSquarespace
\faStackExchange


\faStaylinked
\faStopwatch
\faStopwatch20
\faStore
\faStore*
\faStreetView
\faSuitcaseRolling



\faSync
\faTags
\faTasks
\faTerminal
\faTripadvisor

\faUsersCog
\faUserFriends



\faWallet
\faYahoo


\cvtag{}
\cvtag{}
\cvtag{}


\par \vspace{0.1cm} \end{commentA}








%------------------------------------------------------------------------------------------
%\cvsection{\faCertificate CERTIFICAÇÕES}
%%-------------------------
% Certificações

\cvevent
	{%
		\faGoogle\ Certificações Google ADS%
		\quad% pequeno espaço antes da linha
		\hrulefill%
		}
	{Google Skillshop \hfill \faEdge Treinamento Web}
	{Julho 2023 -- Agosto 2023}
	{Fortaleza, CE, Brasil}
% remove aquele espaço extra logo após o título
\vspace{-0.5em} 
% Ajusta espaçamento e remove “corrida” de itens
\begin{multicols}{2}
	\begin{itemize}[leftmargin=*,itemsep=0.5em,topsep=0.5em]
		\item Anúncios de a performance com tecnologia de IA.
		\item Aumente as vendas off-line.
		\item Rede de pesquisa do Google Ads.
		\item Criativos do Google Ads.
		\item Anúncios do Shopping.
		\item Apps do Google Ads.
		\item Campanhas	de vídeo no Google Ads.
		\item Display do Google Ads.
		\item Certificação em métricas do Google Ads.
	\end{itemize}
\end{multicols}

\cvevent
	{\faFacebook  Certificações Meta ADS%
		\quad% pequeno espaço antes da linha
		\hrulefill%
		}
	{Meta Blueprint \hfill \faEdge Treinamento Web}
	{Agosto 2023 -- Setembro 2023}
	{Fortaleza, CE, Brasil}
% remove aquele espaço extra logo após o título
\vspace{-0.5em} 
% Ajusta espaçamento e remove “corrida” de itens
\begin{multicols}{2}
	\begin{itemize}[leftmargin=*,itemsep=0.5em,topsep=0.5em]
		\item Planeiamento de Mídia.
		\item Compra de Mídia.
		\item Marketing Science.
		\item Estratégia Criativa.
		\item Associado de Marketing Digital.
		\item Gerenciamento de Comunidade.
		\item Estratégia de Marketing de Negócios.
	\end{itemize}
\end{multicols}

\cvevent
{\faMicrosoft Certificados Microsoft Office Specialist Master}
{Microsoft \hfill \faEdge Treinamento Web}
{Junho 2019 -- Agosto 2019}
{Fortaleza, CE, Brasil}
\cvtag
{\faFileExcel[regular] Excel 2019 Expert} 
\cvtag
{\faFileWord[regular] Word 2019 Expert} 
\cvtag
{Outlook 2019} \\ 
\cvtag
{\faDatabase Access 2016} 
\cvtag
{\faFilePowerpoint[regular] PowerPoint 2016}  
\cvtag
{Project 2019}

%------------------------------------------------------------------------------------------
%\cvsection{\faAddressBook[regula] COMPETÊNCIAS\\ \hspace{0.9cm} PESSOAL}
%%-------------------------
\begin{commentA}
\cvsubsection{\faPeopleArrows ~Relacionamento e Colaboração}
\cviconitemize{
  \item[\faUsers] Promove sinergia entre áreas como Operações, Comercial e Comunicação, execução regional integrada
  \item[\faHandshake] Atua como contato entre parceiros e equipes regionais, empatia em ambientes multifuncionais
}

\cvsubsection{\faComments ~Pensamento Estratégico e Liderança}
\cviconitemize{
  \item[\faCogs] Facilita cooperação e iniciativa para propor ações conforme o comportamento local
  \item[\faCompass] Adapta estratégias conforme o contexto de cada praça, foco em melhoria contínua e entrega de valor
}
\end{commentA}
%-------------------------
%-------------------------
\begin{commentB}
  
\end{commentB}


%------------------------------------------------------------------------------------------
%\cvsection{\faGamepad ~HOBBIES}
%

\begin{itemize}
	\item \faPlaystation \faXbox \faSteam ~Jogar videojogos \faGamepad
	\item \faSpotify ~Ouvir música.
	\item \faFingerprint ~Impressão 3D.
	\item \faRoute ~Pilotar Drone.
\end{itemize}

%------------------------------------------------------------------------------------------
%\cvsection{\faMedal HONRA}
%
\cvachievement{\faHeartbeat}{Implementação de sistema}{Participação na implementação de uma nova funcionabilidade de preferencia de região no aplicativo Uber para motorista.}

%Prêmios e menções honrosas
%Grand Prix Bahia Recall 2010 – Categoria Internet
%Vencedor do Grand Prix na categoria Internet do Bahia Recall 2010. Peça: Paz no Trânsito || Agência: Ideia 3 || Cliente: Prefeitura Municipal de Salvador. [peça]

%Referência Nacional em Mídias Sociais
%Citado por André Telles, CEO da agência de marketing digital Mentes Digitais, como referência nacional em mídias sociais no livro “A Revolução das Mídias Sociais”.

%------------------------------------------------------------------------------------------
%\cvsection{Referências}
%

% \cvref{name}{email}{mailing address}
\cvref{\faUniversity ~Prof.\ Alpha Beta}{Institute}{a.beta@university.edu}
{Address Line 1\\Address line 2}

\divider

\cvref{Prof.\ Gamma Delta}{Institute}{g.delta@university.edu}
{Address Line 1\\Address line 2}


\end{paracol}

%------------------------------------------------------------------------------------------
% Start a 1-column paracol.
\begin{paracol}{1}
	\includegraphics[scale=0.95]{Carta Alloha.pdf}\\
\end{paracol}

%------------------------------------------------------------------------------------------
% Start a 1-column paracol.
\begin{paracol}{1}
	\includegraphics[scale=0.95]{/home/dns/Downloads/Analista_de_Marketing/Conteúdo/Imagem/Carta de recomendaçãoF.pdf}\\
\end{paracol}
%------------------------------------------------------------------------------------------

% use ONLY \newpage if you want to force a page break for
% ONLY the currentc column
%\newpage
%% Set the left/right column width ratio to 1:0.
%\columnratio{1.0}

\end{document}
